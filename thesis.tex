 %章节数字字号
%%=======================================
\documentclass[cs4size,a4paper,UTF8]{ctexart}   
\usepackage{setspace}
%==================== 字体支持 ============
\setCJKfamilyfont{zhsong}{STZhongsong}
\newcommand{\zhsong}{\CJKfamily{zhsong}} % 华文中体
\setCJKfamilyfont{tnr}{Times New Roman}
\newcommand{\tnr}{\CJKfamily{tnr}} % Times New Roman
\setCJKfamilyfont{myArial}{Arial}
\newcommand{\Arial}{\CJKfamily{myArial}} %定义字体Arial

\renewcommand*{\songti}{\CJKfamily{SimSun}} %定义字体宋体
\renewcommand*{\heiti}{\CJKfamily{SimHei}} %定义字体黑体
%==================== 符号支持 ============
\usepackage{pifont}
\usepackage{url}
%带圈数字 从172起
%==================== 空白页命令============
\newcommand\blankpage{
	\newpage
	~\thispagestyle{empty}
	\newpage
}
%==================== 数学符号公式 ============
\usepackage{amsmath}                 % AMS LaTeX宏包
\usepackage{amssymb}                 %用来排版漂亮的数学公式
\usepackage{amsbsy}
\usepackage{bm}  %公式加粗
\usepackage[style=1]{mdframed}
\usepackage{amsthm}
\usepackage{enumitem}
\usepackage{amsfonts}
\usepackage{mathrsfs}                % 英文花体字 体
\usepackage{bm}                      % 数学公式中的黑斜体
\usepackage{bbding,manfnt}           % 一些图标,如 \dbend
\usepackage{lettrine}                % 首字下沉,命令\lettrine
\def\attention{\lettrine[lines=2,lraise=0,nindent=0em]{\large\textdbend\hspace{1mm}}{}}
\usepackage{longtable}
\usepackage[toc,page]{appendix}

\usepackage[
a4paper,
top=3.7cm,bottom=3.55cm,left=2.5cm,right=2.0cm,
% headheight=1cm,
headsep=4mm,
footskip=1.1cm,
% showframe,
]{geometry}% 页边距调整

%\usepackage{relsize}                % 调整公式字体大小:\mathsmaller,\mathlarger
%\usepackage{caption2}               % 浮动图形和表格标题样式
%====================公式按章编号==========================
\numberwithin{equation}{section}
\numberwithin{table}{section}
\numberwithin{figure}{section}
%================= 基本格式预置 ===========================
\setmainfont{Times New Roman}%设置默认应为字体

% 文档中文主字体是宋体,对应的 bfseries 字体是黑体
\setCJKmainfont[BoldFont=SimHei]{SimSun} 

% \setCJKsansfont[BoldFont=STHeiti]{STXihei}
% \setCJKmonofont{STFangsong}

% \renewcommand{\baselinestretch}{1.5}
% \setlength{\baselineskip}{20pt}%文字行距为“固定值”20磅

\usepackage{fancyhdr}
\pagestyle{fancy}
\fancyhf{}  
\fancyhead[C]{\zihao{5} 武汉理工大学毕业设计(论文)}%页眉
\fancyfoot[C]{~\tnr\zihao{5} \thepage~}  %页脚
\renewcommand{\headrulewidth}{0.65pt}  %页眉分割线

\CTEXsetup[
		format={\centering\bfseries\zihao{-2}},
		name={第, 章},
		beforeskip = 9pt, %段前、段后0.5行
		afterskip = 9pt,
		% fixbeforeskip = true,
		]{section}
\CTEXsetup[
		format={\bfseries\zihao{3}},
		beforeskip = 8pt, %段前、段后0.5行
		afterskip = 8pt,
		]{subsection}
\CTEXsetup[
		format={\bfseries\zihao{4}},
		beforeskip = 7pt, %段前、段后0.5行
		afterskip = 7pt,
		]{subsubsection}
% \linespread{1.36}\selectfont
% \setstretch{1.36}
%================== 图形支持宏包 =========================
\usepackage{subfigure}
\usepackage{float}
\usepackage{graphicx}                % 嵌入png图像
\usepackage{lscape}
\usepackage{color,xcolor}            % 支持彩色文本、底色、文本框等
\usepackage{hyperref}                % 交叉引用
\usepackage{caption}
\usepackage{array}
\captionsetup{figurewithin=section}
% 图表插入式样预置 
% 空格 space;点 period;冒号 colon
\DeclareCaptionLabelSeparator{twospace}{\ ~}
\captionsetup{labelsep=twospace}                        %去掉图1.1:后冒号
% \renewcommand{\figurename}{Fig.}%图标号由figure改为fig
\captionsetup[table]{labelsep=twospace}
%==================== 源码和流程图 =====================
\usepackage{listings}                % 粘贴源代码
\usepackage{tikz}                    
\usepackage{tikz-3dplot}
\usetikzlibrary{shapes,arrows,positioning}

%================= 目录式样预置 ===========================
\usepackage[subfigure]{tocloft}
\hypersetup{hidelinks} %取消方框
\renewcommand{\cftdot}{.}          % 引导点
\renewcommand{\cftdotsep}{0}        % 引导点的距离
\renewcommand{\cftsecleader}{\cftdotfill{\cftdotsep}}  % 节的引导点设置
\renewcommand{\cftsecfont}{\zihao{-4}}                  % 节项目字体
	\setlength{\cftbeforesecskip}{-2pt} %间距设置
	\setlength{\cftbeforesubsecskip}{-2pt}
  % \setlength{\cftbeforesubsubsecskip}{-2pt}
\renewcommand{\cftsecpagefont}{\tnr\zihao{-4}}
% ===================   参考文献预置    ===================
% % \usepackage{biblatex}
\bibliographystyle{chinesebst}
\newcommand{\upcite}[1]{\textsuperscript{\textsuperscript{\cite{#1}}}} %引用上标
\usepackage{cite}
% --------参考文献间距调整-------
% \usepackage{bibspacing}
% \setlength{\bibitemsep}{0.2\baselineskip plus 0.05\baselineskip minus .05\baselineskip}
\newlength{\bibitemsep}\setlength{\bibitemsep}{0.02\baselineskip plus 0.05\baselineskip minus 0.05\baselineskip}
\newlength{\bibparskip}\setlength{\bibparskip}{3pt}
\let\oldthebibliography\thebibliography
\renewcommand\thebibliography[1]{%
  \oldthebibliography{#1}%
  \setlength{\parskip}{\bibitemsep}%
  \setlength{\itemsep}{\bibparskip}%
}
% \usepackage{cite}

% 4.一幅大图里面想要实现4个小图,且每个图都有标号(比如a,b,c,d)和注释?
% \begin{figure}[htb]
% \centering
% \includegraphics[width=2.9cm]{1.eps}
% \includegraphics[width=2.9cm]{2.eps}
% \includegraphics[width=2.9cm]{3.eps}
% \includegraphics[width=2.9cm]{4.eps}
% \put(-305,-10){\footnotesize{(a)}}
% \put(-220,-10){\footnotesize{(b)}}
% \put(-135,-10){\footnotesize{(c)}} 
% \put(-45,-10){\footnotesize{(d)}}
% \caption{xxxxxxxxxxxxxxxxxxxxx: (a)
% xxxxxxx; (b) xxxxxxxxxx; (c) xxxxxxxxxx; (d)xxxxxxxxx.} \label{fig1}
% \end{figure}

%===================   正文开始    ===================
\begin{document}
%===================  定理类环境定义 ===================
\newtheorem{example}{例}[section]              % 整体编号
\newtheorem{algorithm}{算法}[section]
\newtheorem{theorem}{\hspace*{2em}定理}[section]            % 按 section 编号
\newtheorem{definition}{\hspace*{2em}定义}[section]
\newtheorem{axiom}{公理}[section]
\newtheorem{property}{性质}[section]
\newtheorem{proposition}{命题}[section]
\newtheorem{lemma}{引理}[section]
\newtheorem{corollary}{推论}[section]
\newtheorem{remark}{注解}[section]
\newtheorem{condition}{条件}[section]
\newtheorem{conclusion}{结论}[section]
\newtheorem{assumption}{假设}[section]
%==================重定义 ===================
\renewcommand{\contentsname}{\hspace*{\fill}\heiti\zihao{-2}目\quad 录\hspace*{\fill}}
\renewcommand{\abstractname}{摘要} 
\renewcommand{\refname}{参考文献}     
\renewcommand{\indexname}{索引}
\renewcommand{\figurename}{图}
\renewcommand{\tablename}{表}
%\renewcommand{\appendixname}{附录}
\renewcommand{\proofname}{证明}
\renewcommand{\algorithm}{算法} 
%============== 封皮和前言 =================
%===============  封面  =================
\begin{center}
% \begin{figure}[!th]
% \centering
% \includegraphics[width=0.7\linewidth]{figure/SchoolName}
% \end{figure}

\vspace*{1.55cm}
{\zhsong\zihao{1} 武汉理工大学毕业设计(论文)} \\
\vspace*{3.5cm}
 {\bfseries\zihao{2} 有损压缩方法研究 }\\
\vspace*{4cm}
\zhsong
\begin{tabular}{cc}
 \zihao{3} 学院(系): &\underline{\makebox[7cm][c]{\zihao{3}计算机科学与技术学院}} \\ 
 \\
 \zihao{3}专业班级: & \underline{\makebox[7cm][c]{\zihao{3}软件工程软件ZY1402班}} \\ 
 \\
 \zihao{3}学生姓名: & \underline{\makebox[7cm][c]{\zihao{3}叶哲宇}} \\ 
 \\
 \zihao{3}指导教师: & \underline{\makebox[7cm][c]{\zihao{3}钟~~~珞}} \\ 
 \\
\end{tabular} 
\end{center}
\thispagestyle{empty}
\clearpage
%=====================原创性声明===========
\begin{center}
\zihao{-2} \bfseries{学位论文原创性声明}
\end{center}
\setstretch{1.8}

本人郑重声明: 所呈交的论文是本人在导师的指导下独立进行研究所取得的研究成果。除了文中特别加以标注引用的内容外,本论文不包括任何其他个人或集体已经发表或撰写的成果作品。本人完全意识到本声明的法律后果由本人承担。 \\
\hspace*{9.2cm} \zihao{-4} 作者签名:\\
\hspace*{10.45cm} 年\qquad 月\qquad 日 
\vskip 4.9cm
\begin{center}
\zihao{-2} \bfseries{学位论文版权使用授权书}
\end{center}
\setstretch{1.8}

本学位论文作者完全了解学校有关保障、使用学位论文的规定,同意学校保留并向有关学位论文管理部门或机构送交论文的复印件和电子版,允许论文被查阅和借阅。本人授权省级优秀学士论文评选机构将本学位论文的全部或部分内容编入有关数据进行检索,可以采用影印、缩印或扫描等复制手段保存和汇编本学位论文。

\begin{tabular}[t]{l}
1、保密$ \Box$,在~~~年解密后适用本授权书  \\ 
2、不保密$ \Box$  \\ 
(请在以上相应方框内打“$\surd”$)
\end{tabular} \\
\zihao{-4} 
\hspace*{9.2cm}作者签名:   \\
\hspace*{10.45cm}年 \quad  月  \quad  日\\
\hspace*{9.2cm}导师签名:    \\
\hspace*{10.45cm}年 \quad  月 \quad   日\\

\thispagestyle{empty}
\clearpage

%%=============设计(论文)任务书===========
%\begin{center}
%\zihao{-2}\textbf{\songti 本科生毕业设计(论文)任务书} 
%\end{center}
%\smallskip
%\renewcommand{\arraystretch}{1.3}
%\begin{tabular}{lll}
%\zihao{4} \textbf{\songti 学生姓名: 曹宇} & & \zihao{4} \textbf{\songti 专业班级:\quad\quad 船海1006班} \\ 
%\zihao{4} \textbf{\songti 指导教师:徐海祥}&\makebox [3cm] & \zihao{4} \textbf{\songti 工作单位:\quad 武汉理工大学} \\ 
%\end{tabular}\\
%\begin{tabular}{lll}
%\zihao{4} \textbf{\songti 设计(论文)题目:}& \zihao{4} \textbf{\songti  武汉理工本科论文\LaTeX 模板 } &\\ 
%\zihao{4} \textbf{\songti 设计(论文)主要内容:} \\
%\end{tabular} \\ 
%\begin{enumerate}
%\item \LaTeX 环境的配置
%\item 主要字体的控制和数学公式的选用
%\item 图表和代码的粘贴
%\end{enumerate}
%\begin{tabular}{ll}
%\zihao{4} \textbf{\songti 要求完成的主要任务:}
%\end{tabular} \\ 
%\begin{enumerate}
%\item 选择合适的\TeX 编辑系统
%\item 学习如何使用控制代码完成排版
%\item 合理的安排学习和科研的时间来发展自己兴趣爱好
%\end{enumerate}
%\begin{tabular}{ll}
%\zihao{4} \textbf{\songti 必读参考资料:}
%\end{tabular}
%\begin{enumerate}
%\item \LaTeX  \quad User Manual
%\item  字体设计的艺术
%\end{enumerate}
%\begin{tabular}{lll}
%\zihao{4} \textbf{\songti 指导教师签名: }&\makebox [4cm]& \zihao{4} \textbf{\songti 系主任签名:} \\
%& & \zihao{4} \textbf{\songti 院长签名(章)}
%\end{tabular}
%\thispagestyle{empty}
%\clearpage
%%==========本科生毕业设计(论文)开题报告  =============
%\begin{center}
%\zihao{-2} \textbf{\songti 武汉理工大学}\\
%\zihao{-2} \textbf{\songti 本科生毕业设计(论文)开题报告} 
%\end{center}
%\begin{tabular}{|l|}
%\hline \rule[-2ex]{0pt}{5.5ex} \makebox[13.5cm][l]{\zihao{4} \heiti 1、目的及意义(含国内外的研究现状分析) } \\ 
%\quad \LaTeX 是国际通行的科技论文排版软件,国际上科研机构和大学都采用它写作\\
%\quad 国内著名高校都有自己的本科生\LaTeX 模板供毕业生使用\\
%\quad 但是武汉理工大学还没有本科生\LaTeX 模板可以参考\\
%\quad 人类的价值在于创造而不是索取 \\
%\hline \rule[-2ex]{0pt}{5.5ex}  \zihao{4} \heiti
%2、基本内容和技术方案\\ 
%\quad 采用GITHUB托管降低代码维护成本\\
%\quad 加入在线\TeX 编辑器的使用简介 \\
%\quad 授人以渔,注重方法和理念的引导\\
%\hline \rule[-2ex]{0pt}{5.5ex}  \zihao{4} \heiti
%3、进度安排 \\ 
%\quad 离 deadline 两个月吃喝玩乐 \\
%\quad 离 deadline 一个月吃喝玩乐 \\
%\quad 离 deadline 半个月吃喝玩乐 \\
%\quad 离 deadline 一个星期狂写论文 \\
%\hline \rule[-2ex]{0pt}{5.5ex} \zihao{4} \heiti
%4、指导教师意见 \\ 
%\quad 曹宇同学是个好同志\\
%\quad 曹宇同志是个好同学\\
%\quad 本表格是支持跨页的长表格,你可以复制上面的内容进行测试\\
%\quad 具体方法是将tabular改为 longtable然后再编译\\
%\makebox[10cm][r]指导教师签名:\\
%\makebox[12cm][r]\quad 年\quad 月\quad 日\\
%\hline 
%\end{tabular} 
%\thispagestyle{empty}


%\blankpage
\pagestyle{plain}
\pagenumbering{Roman}
\setstretch{1.36} %12*1.2*1.36=20
\section*{\zihao{2} \centering 摘要}

\vskip0.5cm
本文基于武汉理工大学本科生毕业论文格式2018年的相关要求,结合\LaTeX 在实际运用中的基本技巧和方法对于科技论文排版方法进行一个简略的介绍。通过参照本科生毕业论文的相关要求,实现了符合国际科技论文排版规则的具有一定美感的毕业论文模板设计。 


\textbf{关键词:}  武汉理工大学, 毕业论文 
\addcontentsline{toc}{section}{摘要}

\clearpage
\section*{\zihao{2} \centering \textbf{Abstract} }
   %用了Times New Roman字体来美化观感

In this short article we will discuss about \LaTeX\,  for your dissertation \\

\textbf{Key Words:} WHUT, Bachelor Thesis
\addcontentsline{toc}{section}{Abstract}





\pagestyle{empty}
\tableofcontents  %目录
% \pdfbookmark{目录}{Contents} %在pdf中显示目录链接
% \thispagestyle{empty} 第二页没有页码
%============== 论文正文   =================
\pagestyle{fancy}
\pagenumbering{arabic}
\section{\LaTeX 入门简介}
\LaTeX 是国际通行的格式化排版系统,在数学界和计算机科学界有着极为广泛的运用。学习\LaTeX 排版规则是每一个科研人员熟悉科研论文格式化写作,提高论文质量的不二之选\cite{lamport1994latex}。
\subsection{编辑环境}
编译环境由编辑器和编译器两个部分组成, 编辑器的功能和我们常见的写字板差不多,它能够了方便我们处理\TeX 源码明确彼此之间的篇章关系,从而提高排版效率。而编译器则是将\TeX 语言转化为计算机能够理解的二进制代码并最终呈现为我们能够阅读的PDF文档,他们之间相互分工共同完成排版任务。
\subsubsection{编辑器}
编辑器的种类非常多,有“所见即所得”的\textbf{LyX},也有Linux向的\textbf{Emacs}和\textbf{Vim},还有伪geek向的\href{http://www.sublimetext.com/}{\textbf{Sublime Text}},而我自己则偏爱IDE向的
\href{http://texstudio.sourceforge.net/}{\textbf{\TeX Studio}}.它有着一些令我爱不释手的特性,如:
\begin{enumerate}
\item 清晰的组织结构,你可以在屏幕左侧看到他们
\item 便捷的自动补充功能,只要输入命令的一部分就能够完成撰写
\item 合理的宏包查看方式,右键菜单中可以找到宏包的文档
\item 贴心的实用工具,矩阵插入助手,表格编辑助手等
\end{enumerate}

每个人都可以选择自己顺手的编辑器,如果你真的非常懒不愿意在如此多的选择中做出一个抉择那么编译器中自带的\textbf{\TeX Works}也是一个不错的选择。
\subsubsection{编译器}
编译器一般存在于封装了宏包的各种\TeX 发行包中,按照宏包数量的多少从几十兆字节到若干个G都有。按照操作系统平台的不同,比较流行的发行包有\TeX Live,pro\TeX t和Mac\TeX . 在Windows 平台或者 Linux 平台上常用的是\href{https://www.tug.org/texlive/}{\TeX Live},如果您需要从网络上下载请选择\href{https://www.tug.org/texlive/acquire-iso.html}{ISO镜像}进行下载。国内知名大学均有镜像FTP下载站,通过他们你可以获得这个3GB左右的ISO包,安装它可以免去您下载各类宏包和寻找文档的麻烦。
\subsection{尝试编译}
\subsubsection{Windows}
安装并设置完毕软件环境之后,就可以尝试对于本论文进行编译工作。打开文件夹中的\verb|thesis.tex| 文件,将默认编译器设置为Xe\LaTeX(\TeX Studio 中依次点击Options - Configure TeXstudio - Built - Default Complier 内选择Xe\LaTeX ,\TeX works 则可以选择左上角的下拉菜单在其中找到Xe\LaTeX ),点击编译按钮就可以开始编译过程了。

正常编译结束之后,文件夹中会出现一个\verb|thesis.pdf|的文件同时编辑器也会自动打开该文件生成一个精美的预览。你可以对比自己编译出来的成果与本文件之间的差异,来确定编译器和编辑器是否已经设置妥当。
\subsubsection{Mac OS X}

在OS X系统下,由于系统内字体的区别,本模板会遇到一些编译上的问题。 我们需要手动调整一下字体的设置,以正常编译模板, 具体修改方式可以参见\href{http://www.zhihu.com/question/22906637}{知乎问答}。 

问答的第四步可能需要一些修改,
\begin{verbatim}
	cd /usr/local/texlive/2014/texmf-dist/tex/texlive/ctex
\end{verbatim}

\subsubsection{Linux}
本模板在Ubuntu 14.04 以及12.04 长期稳定支持版上均编译通过。

\subsection{简单步骤}
先\href{https://www.tug.org/texlive/acquire-iso.html}{下载}\TeX 发行包(内含编译器和相关宏包及文档),安装这个发行包大概需要20分钟左右的时间,安装期间请关闭杀毒软件以保证组件的顺利注册。
使用自带的编辑器或者下载\href{http://texstudio.sourceforge.net/}{\textbf{\TeX Studio}},作为默认编辑器使用。打开\verb|thesis.tex|,并设置编译器为Xe\LaTeX 再进行一次编译。如果遇到无法编译的问题请注意以下技术细节:

相关路径设置是否正确,在\textbf{\TeX Studio}的Options - Configure TeXstudio - Commands 中检查路径,正确的路径形式应该类似于

\begin{verbatim}
"D:/Program Files/texlive/2013/bin/win32/latex.exe"
 -src -interaction=nonstopmode %.tex
\end{verbatim}

























      %
\include{body/chapter2}
\include{body/chapter3}
\include{body/chapter4}
\include{body/chapter5}
\include{body/chapter6}
%\include{body/chapter7}
%============= 参考文献 =====================
% \begin{thebibliography}{99}
\setstretch{1.51} %11
\addcontentsline{toc}{section}{参考文献}
\zihao{5}
%\fontsize{10.5pt}{20pt}\selectfon
\bibliography{bibfile}
% \setlength{\bibsep}{3pt}
% \end{thebibliography}
\clearpage
%\newpage
\section*{附录}
\addcontentsline{toc}{section}{附录}
\appendix
 %附录
%=============  致谢  ======================
\section*{致谢}
\addcontentsline{toc}{section}{致谢}

在tsaoyu本科生论文的基础完成了最新2018年格式要求的修订,其中论文引用格式GBT7714-2005-BibTeX-Style是上海财经大学的Haixing Hu作品,本模板离不开这些有益的资源的支持。同样感谢正在使用这个模板的你,相信通过你们的使用和传播,这个模板会变得越来越完善。
\end{document}
%%%%%%%%%% 结束 %%%%%%%%%%